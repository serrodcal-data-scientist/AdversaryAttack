
% Default to the notebook output style

    


% Inherit from the specified cell style.




    
\documentclass[11pt]{article}

    
    
    \usepackage[T1]{fontenc}
    % Nicer default font (+ math font) than Computer Modern for most use cases
    \usepackage{mathpazo}

    % Basic figure setup, for now with no caption control since it's done
    % automatically by Pandoc (which extracts ![](path) syntax from Markdown).
    \usepackage{graphicx}
    % We will generate all images so they have a width \maxwidth. This means
    % that they will get their normal width if they fit onto the page, but
    % are scaled down if they would overflow the margins.
    \makeatletter
    \def\maxwidth{\ifdim\Gin@nat@width>\linewidth\linewidth
    \else\Gin@nat@width\fi}
    \makeatother
    \let\Oldincludegraphics\includegraphics
    % Set max figure width to be 80% of text width, for now hardcoded.
    \renewcommand{\includegraphics}[1]{\Oldincludegraphics[width=.8\maxwidth]{#1}}
    % Ensure that by default, figures have no caption (until we provide a
    % proper Figure object with a Caption API and a way to capture that
    % in the conversion process - todo).
    \usepackage{caption}
    \DeclareCaptionLabelFormat{nolabel}{}
    \captionsetup{labelformat=nolabel}

    \usepackage{adjustbox} % Used to constrain images to a maximum size 
    \usepackage{xcolor} % Allow colors to be defined
    \usepackage{enumerate} % Needed for markdown enumerations to work
    \usepackage{geometry} % Used to adjust the document margins
    \usepackage{amsmath} % Equations
    \usepackage{amssymb} % Equations
    \usepackage{textcomp} % defines textquotesingle
    % Hack from http://tex.stackexchange.com/a/47451/13684:
    \AtBeginDocument{%
        \def\PYZsq{\textquotesingle}% Upright quotes in Pygmentized code
    }
    \usepackage{upquote} % Upright quotes for verbatim code
    \usepackage{eurosym} % defines \euro
    \usepackage[mathletters]{ucs} % Extended unicode (utf-8) support
    \usepackage[utf8x]{inputenc} % Allow utf-8 characters in the tex document
    \usepackage{fancyvrb} % verbatim replacement that allows latex
    \usepackage{grffile} % extends the file name processing of package graphics 
                         % to support a larger range 
    % The hyperref package gives us a pdf with properly built
    % internal navigation ('pdf bookmarks' for the table of contents,
    % internal cross-reference links, web links for URLs, etc.)
    \usepackage{hyperref}
    \usepackage{longtable} % longtable support required by pandoc >1.10
    \usepackage{booktabs}  % table support for pandoc > 1.12.2
    \usepackage[inline]{enumitem} % IRkernel/repr support (it uses the enumerate* environment)
    \usepackage[normalem]{ulem} % ulem is needed to support strikethroughs (\sout)
                                % normalem makes italics be italics, not underlines
    

    
    
    % Colors for the hyperref package
    \definecolor{urlcolor}{rgb}{0,.145,.698}
    \definecolor{linkcolor}{rgb}{.71,0.21,0.01}
    \definecolor{citecolor}{rgb}{.12,.54,.11}

    % ANSI colors
    \definecolor{ansi-black}{HTML}{3E424D}
    \definecolor{ansi-black-intense}{HTML}{282C36}
    \definecolor{ansi-red}{HTML}{E75C58}
    \definecolor{ansi-red-intense}{HTML}{B22B31}
    \definecolor{ansi-green}{HTML}{00A250}
    \definecolor{ansi-green-intense}{HTML}{007427}
    \definecolor{ansi-yellow}{HTML}{DDB62B}
    \definecolor{ansi-yellow-intense}{HTML}{B27D12}
    \definecolor{ansi-blue}{HTML}{208FFB}
    \definecolor{ansi-blue-intense}{HTML}{0065CA}
    \definecolor{ansi-magenta}{HTML}{D160C4}
    \definecolor{ansi-magenta-intense}{HTML}{A03196}
    \definecolor{ansi-cyan}{HTML}{60C6C8}
    \definecolor{ansi-cyan-intense}{HTML}{258F8F}
    \definecolor{ansi-white}{HTML}{C5C1B4}
    \definecolor{ansi-white-intense}{HTML}{A1A6B2}

    % commands and environments needed by pandoc snippets
    % extracted from the output of `pandoc -s`
    \providecommand{\tightlist}{%
      \setlength{\itemsep}{0pt}\setlength{\parskip}{0pt}}
    \DefineVerbatimEnvironment{Highlighting}{Verbatim}{commandchars=\\\{\}}
    % Add ',fontsize=\small' for more characters per line
    \newenvironment{Shaded}{}{}
    \newcommand{\KeywordTok}[1]{\textcolor[rgb]{0.00,0.44,0.13}{\textbf{{#1}}}}
    \newcommand{\DataTypeTok}[1]{\textcolor[rgb]{0.56,0.13,0.00}{{#1}}}
    \newcommand{\DecValTok}[1]{\textcolor[rgb]{0.25,0.63,0.44}{{#1}}}
    \newcommand{\BaseNTok}[1]{\textcolor[rgb]{0.25,0.63,0.44}{{#1}}}
    \newcommand{\FloatTok}[1]{\textcolor[rgb]{0.25,0.63,0.44}{{#1}}}
    \newcommand{\CharTok}[1]{\textcolor[rgb]{0.25,0.44,0.63}{{#1}}}
    \newcommand{\StringTok}[1]{\textcolor[rgb]{0.25,0.44,0.63}{{#1}}}
    \newcommand{\CommentTok}[1]{\textcolor[rgb]{0.38,0.63,0.69}{\textit{{#1}}}}
    \newcommand{\OtherTok}[1]{\textcolor[rgb]{0.00,0.44,0.13}{{#1}}}
    \newcommand{\AlertTok}[1]{\textcolor[rgb]{1.00,0.00,0.00}{\textbf{{#1}}}}
    \newcommand{\FunctionTok}[1]{\textcolor[rgb]{0.02,0.16,0.49}{{#1}}}
    \newcommand{\RegionMarkerTok}[1]{{#1}}
    \newcommand{\ErrorTok}[1]{\textcolor[rgb]{1.00,0.00,0.00}{\textbf{{#1}}}}
    \newcommand{\NormalTok}[1]{{#1}}
    
    % Additional commands for more recent versions of Pandoc
    \newcommand{\ConstantTok}[1]{\textcolor[rgb]{0.53,0.00,0.00}{{#1}}}
    \newcommand{\SpecialCharTok}[1]{\textcolor[rgb]{0.25,0.44,0.63}{{#1}}}
    \newcommand{\VerbatimStringTok}[1]{\textcolor[rgb]{0.25,0.44,0.63}{{#1}}}
    \newcommand{\SpecialStringTok}[1]{\textcolor[rgb]{0.73,0.40,0.53}{{#1}}}
    \newcommand{\ImportTok}[1]{{#1}}
    \newcommand{\DocumentationTok}[1]{\textcolor[rgb]{0.73,0.13,0.13}{\textit{{#1}}}}
    \newcommand{\AnnotationTok}[1]{\textcolor[rgb]{0.38,0.63,0.69}{\textbf{\textit{{#1}}}}}
    \newcommand{\CommentVarTok}[1]{\textcolor[rgb]{0.38,0.63,0.69}{\textbf{\textit{{#1}}}}}
    \newcommand{\VariableTok}[1]{\textcolor[rgb]{0.10,0.09,0.49}{{#1}}}
    \newcommand{\ControlFlowTok}[1]{\textcolor[rgb]{0.00,0.44,0.13}{\textbf{{#1}}}}
    \newcommand{\OperatorTok}[1]{\textcolor[rgb]{0.40,0.40,0.40}{{#1}}}
    \newcommand{\BuiltInTok}[1]{{#1}}
    \newcommand{\ExtensionTok}[1]{{#1}}
    \newcommand{\PreprocessorTok}[1]{\textcolor[rgb]{0.74,0.48,0.00}{{#1}}}
    \newcommand{\AttributeTok}[1]{\textcolor[rgb]{0.49,0.56,0.16}{{#1}}}
    \newcommand{\InformationTok}[1]{\textcolor[rgb]{0.38,0.63,0.69}{\textbf{\textit{{#1}}}}}
    \newcommand{\WarningTok}[1]{\textcolor[rgb]{0.38,0.63,0.69}{\textbf{\textit{{#1}}}}}
    
    
    % Define a nice break command that doesn't care if a line doesn't already
    % exist.
    \def\br{\hspace*{\fill} \\* }
    % Math Jax compatability definitions
    \def\gt{>}
    \def\lt{<}
    % Document parameters
    \title{Adversary Attack}
    
    
    

    % Pygments definitions
    
\makeatletter
\def\PY@reset{\let\PY@it=\relax \let\PY@bf=\relax%
    \let\PY@ul=\relax \let\PY@tc=\relax%
    \let\PY@bc=\relax \let\PY@ff=\relax}
\def\PY@tok#1{\csname PY@tok@#1\endcsname}
\def\PY@toks#1+{\ifx\relax#1\empty\else%
    \PY@tok{#1}\expandafter\PY@toks\fi}
\def\PY@do#1{\PY@bc{\PY@tc{\PY@ul{%
    \PY@it{\PY@bf{\PY@ff{#1}}}}}}}
\def\PY#1#2{\PY@reset\PY@toks#1+\relax+\PY@do{#2}}

\expandafter\def\csname PY@tok@w\endcsname{\def\PY@tc##1{\textcolor[rgb]{0.73,0.73,0.73}{##1}}}
\expandafter\def\csname PY@tok@c\endcsname{\let\PY@it=\textit\def\PY@tc##1{\textcolor[rgb]{0.25,0.50,0.50}{##1}}}
\expandafter\def\csname PY@tok@cp\endcsname{\def\PY@tc##1{\textcolor[rgb]{0.74,0.48,0.00}{##1}}}
\expandafter\def\csname PY@tok@k\endcsname{\let\PY@bf=\textbf\def\PY@tc##1{\textcolor[rgb]{0.00,0.50,0.00}{##1}}}
\expandafter\def\csname PY@tok@kp\endcsname{\def\PY@tc##1{\textcolor[rgb]{0.00,0.50,0.00}{##1}}}
\expandafter\def\csname PY@tok@kt\endcsname{\def\PY@tc##1{\textcolor[rgb]{0.69,0.00,0.25}{##1}}}
\expandafter\def\csname PY@tok@o\endcsname{\def\PY@tc##1{\textcolor[rgb]{0.40,0.40,0.40}{##1}}}
\expandafter\def\csname PY@tok@ow\endcsname{\let\PY@bf=\textbf\def\PY@tc##1{\textcolor[rgb]{0.67,0.13,1.00}{##1}}}
\expandafter\def\csname PY@tok@nb\endcsname{\def\PY@tc##1{\textcolor[rgb]{0.00,0.50,0.00}{##1}}}
\expandafter\def\csname PY@tok@nf\endcsname{\def\PY@tc##1{\textcolor[rgb]{0.00,0.00,1.00}{##1}}}
\expandafter\def\csname PY@tok@nc\endcsname{\let\PY@bf=\textbf\def\PY@tc##1{\textcolor[rgb]{0.00,0.00,1.00}{##1}}}
\expandafter\def\csname PY@tok@nn\endcsname{\let\PY@bf=\textbf\def\PY@tc##1{\textcolor[rgb]{0.00,0.00,1.00}{##1}}}
\expandafter\def\csname PY@tok@ne\endcsname{\let\PY@bf=\textbf\def\PY@tc##1{\textcolor[rgb]{0.82,0.25,0.23}{##1}}}
\expandafter\def\csname PY@tok@nv\endcsname{\def\PY@tc##1{\textcolor[rgb]{0.10,0.09,0.49}{##1}}}
\expandafter\def\csname PY@tok@no\endcsname{\def\PY@tc##1{\textcolor[rgb]{0.53,0.00,0.00}{##1}}}
\expandafter\def\csname PY@tok@nl\endcsname{\def\PY@tc##1{\textcolor[rgb]{0.63,0.63,0.00}{##1}}}
\expandafter\def\csname PY@tok@ni\endcsname{\let\PY@bf=\textbf\def\PY@tc##1{\textcolor[rgb]{0.60,0.60,0.60}{##1}}}
\expandafter\def\csname PY@tok@na\endcsname{\def\PY@tc##1{\textcolor[rgb]{0.49,0.56,0.16}{##1}}}
\expandafter\def\csname PY@tok@nt\endcsname{\let\PY@bf=\textbf\def\PY@tc##1{\textcolor[rgb]{0.00,0.50,0.00}{##1}}}
\expandafter\def\csname PY@tok@nd\endcsname{\def\PY@tc##1{\textcolor[rgb]{0.67,0.13,1.00}{##1}}}
\expandafter\def\csname PY@tok@s\endcsname{\def\PY@tc##1{\textcolor[rgb]{0.73,0.13,0.13}{##1}}}
\expandafter\def\csname PY@tok@sd\endcsname{\let\PY@it=\textit\def\PY@tc##1{\textcolor[rgb]{0.73,0.13,0.13}{##1}}}
\expandafter\def\csname PY@tok@si\endcsname{\let\PY@bf=\textbf\def\PY@tc##1{\textcolor[rgb]{0.73,0.40,0.53}{##1}}}
\expandafter\def\csname PY@tok@se\endcsname{\let\PY@bf=\textbf\def\PY@tc##1{\textcolor[rgb]{0.73,0.40,0.13}{##1}}}
\expandafter\def\csname PY@tok@sr\endcsname{\def\PY@tc##1{\textcolor[rgb]{0.73,0.40,0.53}{##1}}}
\expandafter\def\csname PY@tok@ss\endcsname{\def\PY@tc##1{\textcolor[rgb]{0.10,0.09,0.49}{##1}}}
\expandafter\def\csname PY@tok@sx\endcsname{\def\PY@tc##1{\textcolor[rgb]{0.00,0.50,0.00}{##1}}}
\expandafter\def\csname PY@tok@m\endcsname{\def\PY@tc##1{\textcolor[rgb]{0.40,0.40,0.40}{##1}}}
\expandafter\def\csname PY@tok@gh\endcsname{\let\PY@bf=\textbf\def\PY@tc##1{\textcolor[rgb]{0.00,0.00,0.50}{##1}}}
\expandafter\def\csname PY@tok@gu\endcsname{\let\PY@bf=\textbf\def\PY@tc##1{\textcolor[rgb]{0.50,0.00,0.50}{##1}}}
\expandafter\def\csname PY@tok@gd\endcsname{\def\PY@tc##1{\textcolor[rgb]{0.63,0.00,0.00}{##1}}}
\expandafter\def\csname PY@tok@gi\endcsname{\def\PY@tc##1{\textcolor[rgb]{0.00,0.63,0.00}{##1}}}
\expandafter\def\csname PY@tok@gr\endcsname{\def\PY@tc##1{\textcolor[rgb]{1.00,0.00,0.00}{##1}}}
\expandafter\def\csname PY@tok@ge\endcsname{\let\PY@it=\textit}
\expandafter\def\csname PY@tok@gs\endcsname{\let\PY@bf=\textbf}
\expandafter\def\csname PY@tok@gp\endcsname{\let\PY@bf=\textbf\def\PY@tc##1{\textcolor[rgb]{0.00,0.00,0.50}{##1}}}
\expandafter\def\csname PY@tok@go\endcsname{\def\PY@tc##1{\textcolor[rgb]{0.53,0.53,0.53}{##1}}}
\expandafter\def\csname PY@tok@gt\endcsname{\def\PY@tc##1{\textcolor[rgb]{0.00,0.27,0.87}{##1}}}
\expandafter\def\csname PY@tok@err\endcsname{\def\PY@bc##1{\setlength{\fboxsep}{0pt}\fcolorbox[rgb]{1.00,0.00,0.00}{1,1,1}{\strut ##1}}}
\expandafter\def\csname PY@tok@kc\endcsname{\let\PY@bf=\textbf\def\PY@tc##1{\textcolor[rgb]{0.00,0.50,0.00}{##1}}}
\expandafter\def\csname PY@tok@kd\endcsname{\let\PY@bf=\textbf\def\PY@tc##1{\textcolor[rgb]{0.00,0.50,0.00}{##1}}}
\expandafter\def\csname PY@tok@kn\endcsname{\let\PY@bf=\textbf\def\PY@tc##1{\textcolor[rgb]{0.00,0.50,0.00}{##1}}}
\expandafter\def\csname PY@tok@kr\endcsname{\let\PY@bf=\textbf\def\PY@tc##1{\textcolor[rgb]{0.00,0.50,0.00}{##1}}}
\expandafter\def\csname PY@tok@bp\endcsname{\def\PY@tc##1{\textcolor[rgb]{0.00,0.50,0.00}{##1}}}
\expandafter\def\csname PY@tok@fm\endcsname{\def\PY@tc##1{\textcolor[rgb]{0.00,0.00,1.00}{##1}}}
\expandafter\def\csname PY@tok@vc\endcsname{\def\PY@tc##1{\textcolor[rgb]{0.10,0.09,0.49}{##1}}}
\expandafter\def\csname PY@tok@vg\endcsname{\def\PY@tc##1{\textcolor[rgb]{0.10,0.09,0.49}{##1}}}
\expandafter\def\csname PY@tok@vi\endcsname{\def\PY@tc##1{\textcolor[rgb]{0.10,0.09,0.49}{##1}}}
\expandafter\def\csname PY@tok@vm\endcsname{\def\PY@tc##1{\textcolor[rgb]{0.10,0.09,0.49}{##1}}}
\expandafter\def\csname PY@tok@sa\endcsname{\def\PY@tc##1{\textcolor[rgb]{0.73,0.13,0.13}{##1}}}
\expandafter\def\csname PY@tok@sb\endcsname{\def\PY@tc##1{\textcolor[rgb]{0.73,0.13,0.13}{##1}}}
\expandafter\def\csname PY@tok@sc\endcsname{\def\PY@tc##1{\textcolor[rgb]{0.73,0.13,0.13}{##1}}}
\expandafter\def\csname PY@tok@dl\endcsname{\def\PY@tc##1{\textcolor[rgb]{0.73,0.13,0.13}{##1}}}
\expandafter\def\csname PY@tok@s2\endcsname{\def\PY@tc##1{\textcolor[rgb]{0.73,0.13,0.13}{##1}}}
\expandafter\def\csname PY@tok@sh\endcsname{\def\PY@tc##1{\textcolor[rgb]{0.73,0.13,0.13}{##1}}}
\expandafter\def\csname PY@tok@s1\endcsname{\def\PY@tc##1{\textcolor[rgb]{0.73,0.13,0.13}{##1}}}
\expandafter\def\csname PY@tok@mb\endcsname{\def\PY@tc##1{\textcolor[rgb]{0.40,0.40,0.40}{##1}}}
\expandafter\def\csname PY@tok@mf\endcsname{\def\PY@tc##1{\textcolor[rgb]{0.40,0.40,0.40}{##1}}}
\expandafter\def\csname PY@tok@mh\endcsname{\def\PY@tc##1{\textcolor[rgb]{0.40,0.40,0.40}{##1}}}
\expandafter\def\csname PY@tok@mi\endcsname{\def\PY@tc##1{\textcolor[rgb]{0.40,0.40,0.40}{##1}}}
\expandafter\def\csname PY@tok@il\endcsname{\def\PY@tc##1{\textcolor[rgb]{0.40,0.40,0.40}{##1}}}
\expandafter\def\csname PY@tok@mo\endcsname{\def\PY@tc##1{\textcolor[rgb]{0.40,0.40,0.40}{##1}}}
\expandafter\def\csname PY@tok@ch\endcsname{\let\PY@it=\textit\def\PY@tc##1{\textcolor[rgb]{0.25,0.50,0.50}{##1}}}
\expandafter\def\csname PY@tok@cm\endcsname{\let\PY@it=\textit\def\PY@tc##1{\textcolor[rgb]{0.25,0.50,0.50}{##1}}}
\expandafter\def\csname PY@tok@cpf\endcsname{\let\PY@it=\textit\def\PY@tc##1{\textcolor[rgb]{0.25,0.50,0.50}{##1}}}
\expandafter\def\csname PY@tok@c1\endcsname{\let\PY@it=\textit\def\PY@tc##1{\textcolor[rgb]{0.25,0.50,0.50}{##1}}}
\expandafter\def\csname PY@tok@cs\endcsname{\let\PY@it=\textit\def\PY@tc##1{\textcolor[rgb]{0.25,0.50,0.50}{##1}}}

\def\PYZbs{\char`\\}
\def\PYZus{\char`\_}
\def\PYZob{\char`\{}
\def\PYZcb{\char`\}}
\def\PYZca{\char`\^}
\def\PYZam{\char`\&}
\def\PYZlt{\char`\<}
\def\PYZgt{\char`\>}
\def\PYZsh{\char`\#}
\def\PYZpc{\char`\%}
\def\PYZdl{\char`\$}
\def\PYZhy{\char`\-}
\def\PYZsq{\char`\'}
\def\PYZdq{\char`\"}
\def\PYZti{\char`\~}
% for compatibility with earlier versions
\def\PYZat{@}
\def\PYZlb{[}
\def\PYZrb{]}
\makeatother


    % Exact colors from NB
    \definecolor{incolor}{rgb}{0.0, 0.0, 0.5}
    \definecolor{outcolor}{rgb}{0.545, 0.0, 0.0}



    
    % Prevent overflowing lines due to hard-to-break entities
    \sloppy 
    % Setup hyperref package
    \hypersetup{
      breaklinks=true,  % so long urls are correctly broken across lines
      colorlinks=true,
      urlcolor=urlcolor,
      linkcolor=linkcolor,
      citecolor=citecolor,
      }
    % Slightly bigger margins than the latex defaults
    
    \geometry{verbose,tmargin=1in,bmargin=1in,lmargin=1in,rmargin=1in}
    
    

    \begin{document}
    
    
    \maketitle
    
    

    
    \hypertarget{ataque-de-adversario}{%
\section{Ataque de adversario}\label{ataque-de-adversario}}

En este cuaderno, voy a intentar realizar un ataque de adversario sobre
una red neuronal construida utilizando Keras/Tensorflow.

Un ataque de adversario consiste en conseguir una imagen que, sin
parecer manipulada, una red neuronal la clasifica erróneamente. Con
esto, se podría llegar a provocar que, por ejemplo, un coche autónomo al
ver una señal de STOP la identifique como una señal de aumento de
velocidad, provocando una situación peligrosa.

    \hypertarget{evaluaciuxf3n-de-la-imagen}{%
\subsection{Evaluación de la imagen}\label{evaluaciuxf3n-de-la-imagen}}

En primer lugar, voy a realizar una evaluación normal de una imagen de
un gato. Para ello, primero importamos las librerías que se van a
necesitar.

    \begin{Verbatim}[commandchars=\\\{\}]
{\color{incolor}In [{\color{incolor}1}]:} \PY{k+kn}{import} \PY{n+nn}{tensorflow} \PY{k}{as} \PY{n+nn}{tf} \PY{c+c1}{\PYZsh{} Importamos la librería de Tensorflow}
        \PY{k+kn}{import} \PY{n+nn}{keras} \PY{c+c1}{\PYZsh{} Importamos Keras que trabaja sobre Tensor facilitándonos el trabajo}
        \PY{k+kn}{import} \PY{n+nn}{matplotlib}\PY{n+nn}{.}\PY{n+nn}{pyplot} \PY{k}{as} \PY{n+nn}{plt} \PY{c+c1}{\PYZsh{} Importamos Matplotlib para poder mostrar gráficas}
        \PY{k+kn}{import} \PY{n+nn}{numpy} \PY{k}{as} \PY{n+nn}{np} \PY{c+c1}{\PYZsh{} Importamos la librería Numpy para trabajar con matrices}
\end{Verbatim}


    \begin{Verbatim}[commandchars=\\\{\}]
Using TensorFlow backend.

    \end{Verbatim}

    Ahora, se carga el modelo precargado desde Keras. Es decir, no hay que
entrenarlo, ya lo está.

    \begin{Verbatim}[commandchars=\\\{\}]
{\color{incolor}In [{\color{incolor}2}]:} \PY{c+c1}{\PYZsh{} Importamos los modelos entrenados de Keras, en especial la InceptionV3, de Google y para imágenes, así como, }
        \PY{c+c1}{\PYZsh{} decode\PYZus{}predictions que nos servirá luego para saber que respuesta nos ha dado la red neuronal}
        \PY{k+kn}{from} \PY{n+nn}{keras}\PY{n+nn}{.}\PY{n+nn}{applications}\PY{n+nn}{.}\PY{n+nn}{inception\PYZus{}v3} \PY{k}{import} \PY{n}{InceptionV3}\PY{p}{,} \PY{n}{decode\PYZus{}predictions}
        \PY{k+kn}{from} \PY{n+nn}{keras} \PY{k}{import} \PY{n}{backend} \PY{k}{as} \PY{n}{K}
        \PY{k+kn}{from} \PY{n+nn}{keras}\PY{n+nn}{.}\PY{n+nn}{preprocessing} \PY{k}{import} \PY{n}{image} \PY{c+c1}{\PYZsh{} Para trabajar con procesado de imágenes}
\end{Verbatim}


    A continuación, se carga el modelo, en este caso ``Inception V3'', una
red neuronal específica para imágenes de Google.

    \begin{Verbatim}[commandchars=\\\{\}]
{\color{incolor}In [{\color{incolor}3}]:} \PY{n}{iv3} \PY{o}{=} \PY{n}{InceptionV3}\PY{p}{(}\PY{p}{)}
\end{Verbatim}


    Finalmente, se va a utilizar la imagen para probar el modelo.

    \begin{Verbatim}[commandchars=\\\{\}]
{\color{incolor}In [{\color{incolor}4}]:} \PY{c+c1}{\PYZsh{} Cargamos la imagen en una variable como array con un tamaño redimensionado a 299x299}
        \PY{n}{X} \PY{o}{=} \PY{n}{image}\PY{o}{.}\PY{n}{img\PYZus{}to\PYZus{}array}\PY{p}{(}\PY{n}{image}\PY{o}{.}\PY{n}{load\PYZus{}img}\PY{p}{(}\PY{l+s+s2}{\PYZdq{}}\PY{l+s+s2}{./gato.png}\PY{l+s+s2}{\PYZdq{}}\PY{p}{,} \PY{n}{target\PYZus{}size}\PY{o}{=}\PY{p}{(}\PY{l+m+mi}{299}\PY{p}{,}\PY{l+m+mi}{299}\PY{p}{)}\PY{p}{)}\PY{p}{)} 
        
        \PY{c+c1}{\PYZsh{} InceptionV3 utiliza un formato diferente, el rango de intensidades va de \PYZhy{}1 a 1. Vamos a reescalar este rango.}
        \PY{c+c1}{\PYZsh{} Pasamos de [0,256) \PYZhy{}\PYZgt{} [\PYZhy{}1,1]}
        
        \PY{n}{X} \PY{o}{/}\PY{o}{=} \PY{l+m+mi}{255} 
        \PY{n}{X} \PY{o}{\PYZhy{}}\PY{o}{=} \PY{l+m+mf}{0.5} 
        \PY{n}{X} \PY{o}{*}\PY{o}{=} \PY{l+m+mi}{2} 
\end{Verbatim}


    La red neuronal pide un formato especial, un tensor, que tiene unas
dimensiones específicas, es decir, se necesita una dimensión más:

    \begin{Verbatim}[commandchars=\\\{\}]
{\color{incolor}In [{\color{incolor}5}]:} \PY{n}{X} \PY{o}{=} \PY{n}{X}\PY{o}{.}\PY{n}{reshape}\PY{p}{(}\PY{p}{[}\PY{l+m+mi}{1}\PY{p}{,} \PY{n}{X}\PY{o}{.}\PY{n}{shape}\PY{p}{[}\PY{l+m+mi}{0}\PY{p}{]}\PY{p}{,} \PY{n}{X}\PY{o}{.}\PY{n}{shape}\PY{p}{[}\PY{l+m+mi}{1}\PY{p}{]}\PY{p}{,} \PY{n}{X}\PY{o}{.}\PY{n}{shape}\PY{p}{[}\PY{l+m+mi}{2}\PY{p}{]}\PY{p}{]}\PY{p}{)}
\end{Verbatim}


    Ahora, se va a ejecutar la predicción soble la imagen:

    \begin{Verbatim}[commandchars=\\\{\}]
{\color{incolor}In [{\color{incolor}6}]:} \PY{n}{y} \PY{o}{=} \PY{n}{iv3}\PY{o}{.}\PY{n}{predict}\PY{p}{(}\PY{n}{X}\PY{p}{)}
\end{Verbatim}


    Para ejecutar la predicción, se utiliza la librería
``decode\_prediction'' que se ha importado previamente:

    \begin{Verbatim}[commandchars=\\\{\}]
{\color{incolor}In [{\color{incolor}7}]:} \PY{n}{decode\PYZus{}predictions}\PY{p}{(}\PY{n}{y}\PY{p}{)}
\end{Verbatim}


\begin{Verbatim}[commandchars=\\\{\}]
{\color{outcolor}Out[{\color{outcolor}7}]:} [[('n02124075', 'Egyptian\_cat', 0.86809808),
          ('n02123045', 'tabby', 0.073509209),
          ('n02123159', 'tiger\_cat', 0.03579814),
          ('n04589890', 'window\_screen', 0.0029933434),
          ('n02127052', 'lynx', 0.00035754172)]]
\end{Verbatim}
            
    La predicción dice que se trata de un gato egipcio. La imagen utilizada
ha sido la siguiente:

    \begin{Verbatim}[commandchars=\\\{\}]
{\color{incolor}In [{\color{incolor}8}]:} \PY{c+c1}{\PYZsh{} Vamos a reescalar primero el rango de intensidades a como estaba originalmente}
        \PY{n}{X1} \PY{o}{=} \PY{n}{np}\PY{o}{.}\PY{n}{copy}\PY{p}{(}\PY{n}{X}\PY{p}{)}
        
        \PY{n}{X1} \PY{o}{/}\PY{o}{=} \PY{l+m+mi}{2}
        \PY{n}{X1} \PY{o}{+}\PY{o}{=} \PY{l+m+mf}{0.5}
        \PY{n}{X1} \PY{o}{*}\PY{o}{=} \PY{l+m+mi}{255}
\end{Verbatim}


    \begin{Verbatim}[commandchars=\\\{\}]
{\color{incolor}In [{\color{incolor}9}]:} \PY{n}{plt}\PY{o}{.}\PY{n}{imshow}\PY{p}{(}\PY{n}{X1}\PY{p}{[}\PY{l+m+mi}{0}\PY{p}{]}\PY{o}{.}\PY{n}{astype}\PY{p}{(}\PY{n}{np}\PY{o}{.}\PY{n}{uint8}\PY{p}{)}\PY{p}{)}
        \PY{n}{plt}\PY{o}{.}\PY{n}{show}\PY{p}{(}\PY{p}{)}
\end{Verbatim}


    \begin{center}
    \adjustimage{max size={0.9\linewidth}{0.9\paperheight}}{output_17_0.png}
    \end{center}
    { \hspace*{\fill} \\}
    
    \hypertarget{generando-la-imagen-para-realizar-el-ataque-de-adversario}{%
\subsection{Generando la imagen para realizar el ataque de
adversario}\label{generando-la-imagen-para-realizar-el-ataque-de-adversario}}

Ahora se va a generar la nueva imagen, la que provocará que la red
neuronal se equivoque. Esto es, ir provocando que se maximice el error:

    \begin{Verbatim}[commandchars=\\\{\}]
{\color{incolor}In [{\color{incolor}10}]:} \PY{n}{input\PYZus{}layer} \PY{o}{=} \PY{n}{iv3}\PY{o}{.}\PY{n}{layers}\PY{p}{[}\PY{l+m+mi}{0}\PY{p}{]}\PY{o}{.}\PY{n}{input}
         \PY{n}{output\PYZus{}layer} \PY{o}{=} \PY{n}{iv3}\PY{o}{.}\PY{n}{layers}\PY{p}{[}\PY{o}{\PYZhy{}}\PY{l+m+mi}{1}\PY{p}{]}\PY{o}{.}\PY{n}{output}
         
         \PY{c+c1}{\PYZsh{} Vamos a hacer creer a la red que queremos que lo clasifique como un limón (número 951)}
         \PY{n}{target\PYZus{}class} \PY{o}{=} \PY{l+m+mi}{951}
         
         \PY{c+c1}{\PYZsh{} Creamos una función de coste para que la última capa el resultado que muestre sea \PYZsq{}target\PYZus{}class\PYZsq{}}
         \PY{n}{loss} \PY{o}{=} \PY{n}{output\PYZus{}layer}\PY{p}{[}\PY{l+m+mi}{0}\PY{p}{,} \PY{n}{target\PYZus{}class}\PY{p}{]}
\end{Verbatim}


    Ahora es necesario definir el gradiente, para encontrar el mínimo, es
decir, el objetivo:

    \begin{Verbatim}[commandchars=\\\{\}]
{\color{incolor}In [{\color{incolor}11}]:} \PY{c+c1}{\PYZsh{} Nos hace el gráfo entre el valor de entrada y el cálculo del gradiente}
         \PY{n}{grad} \PY{o}{=} \PY{n}{K}\PY{o}{.}\PY{n}{gradients}\PY{p}{(}\PY{n}{loss}\PY{p}{,} \PY{n}{input\PYZus{}layer}\PY{p}{)}\PY{p}{[}\PY{l+m+mi}{0}\PY{p}{]}
         
         \PY{c+c1}{\PYZsh{} Función que nos hace el cálculo del gradiente}
         \PY{n}{optimize\PYZus{}gradient} \PY{o}{=} \PY{n}{K}\PY{o}{.}\PY{n}{function}\PY{p}{(}\PY{p}{[}\PY{n}{input\PYZus{}layer}\PY{p}{,} \PY{n}{K}\PY{o}{.}\PY{n}{learning\PYZus{}phase}\PY{p}{(}\PY{p}{)}\PY{p}{]}\PY{p}{,} \PY{p}{[}\PY{n}{grad}\PY{p}{,} \PY{n}{loss}\PY{p}{]}\PY{p}{)}
\end{Verbatim}


    Una vez hecho esto, falta el bucle donde ejecutar este código:

    \begin{Verbatim}[commandchars=\\\{\}]
{\color{incolor}In [{\color{incolor}12}]:} \PY{n}{adv} \PY{o}{=} \PY{n}{np}\PY{o}{.}\PY{n}{copy}\PY{p}{(}\PY{n}{X}\PY{p}{)}
         
         \PY{n}{cost} \PY{o}{=} \PY{l+m+mf}{0.0}
         
         \PY{k}{while} \PY{n}{cost} \PY{o}{\PYZlt{}} \PY{l+m+mf}{0.95}\PY{p}{:}
             \PY{n}{gr}\PY{p}{,} \PY{n}{cost} \PY{o}{=} \PY{n}{optimize\PYZus{}gradient}\PY{p}{(}\PY{p}{[}\PY{n}{adv}\PY{p}{,} \PY{l+m+mi}{0}\PY{p}{]}\PY{p}{)} \PY{c+c1}{\PYZsh{} Utilizamos la copia que realizamos anteriormente}
             \PY{n}{adv} \PY{o}{+}\PY{o}{=} \PY{n}{gr} \PY{c+c1}{\PYZsh{} Manipulamos los píxeles}
\end{Verbatim}


    Finalmente, se devuelve al rango de intesidades la nueva imagen y se
mostrarla:

    \begin{Verbatim}[commandchars=\\\{\}]
{\color{incolor}In [{\color{incolor}13}]:} \PY{n}{adv} \PY{o}{/}\PY{o}{=} \PY{l+m+mi}{2}
         \PY{n}{adv} \PY{o}{+}\PY{o}{=} \PY{l+m+mf}{0.5}
         \PY{n}{adv} \PY{o}{*}\PY{o}{=} \PY{l+m+mi}{255}
\end{Verbatim}


    \begin{Verbatim}[commandchars=\\\{\}]
{\color{incolor}In [{\color{incolor}14}]:} \PY{n}{plt}\PY{o}{.}\PY{n}{imshow}\PY{p}{(}\PY{n}{adv}\PY{p}{[}\PY{l+m+mi}{0}\PY{p}{]}\PY{o}{.}\PY{n}{astype}\PY{p}{(}\PY{n}{np}\PY{o}{.}\PY{n}{uint8}\PY{p}{)}\PY{p}{)}
         \PY{n}{plt}\PY{o}{.}\PY{n}{show}\PY{p}{(}\PY{p}{)}
\end{Verbatim}


    \begin{center}
    \adjustimage{max size={0.9\linewidth}{0.9\paperheight}}{output_26_0.png}
    \end{center}
    { \hspace*{\fill} \\}
    
    Ahora se tiene la imagen hackeada, y se ven ciertos pixeles que pueden
hacer sospechar. Además, se va a guardar dicha imagen para que se
persista el resultado:

    \begin{Verbatim}[commandchars=\\\{\}]
{\color{incolor}In [{\color{incolor}15}]:} \PY{c+c1}{\PYZsh{} Guardamos la imagen}
         \PY{k+kn}{from} \PY{n+nn}{PIL} \PY{k}{import} \PY{n}{Image}
         \PY{n}{im} \PY{o}{=} \PY{n}{Image}\PY{o}{.}\PY{n}{fromarray}\PY{p}{(}\PY{n}{adv}\PY{p}{[}\PY{l+m+mi}{0}\PY{p}{]}\PY{o}{.}\PY{n}{astype}\PY{p}{(}\PY{n}{np}\PY{o}{.}\PY{n}{uint8}\PY{p}{)}\PY{p}{)}
         \PY{n}{im}\PY{o}{.}\PY{n}{save}\PY{p}{(}\PY{l+s+s2}{\PYZdq{}}\PY{l+s+s2}{./gato\PYZus{}hacked.png}\PY{l+s+s2}{\PYZdq{}}\PY{p}{)}
\end{Verbatim}


    Se carga de nuevo la imagen y se realiza la predicción (se espera que el
resultado sea: limón):

    \begin{Verbatim}[commandchars=\\\{\}]
{\color{incolor}In [{\color{incolor}16}]:} \PY{c+c1}{\PYZsh{} Cargamos la imagen en una variable como array con un tamaño redimensionado a 299x299}
         \PY{n}{img\PYZus{}hacked} \PY{o}{=} \PY{n}{image}\PY{o}{.}\PY{n}{img\PYZus{}to\PYZus{}array}\PY{p}{(}\PY{n}{image}\PY{o}{.}\PY{n}{load\PYZus{}img}\PY{p}{(}\PY{l+s+s2}{\PYZdq{}}\PY{l+s+s2}{./gato\PYZus{}hacked.png}\PY{l+s+s2}{\PYZdq{}}\PY{p}{)}\PY{p}{)} 
         
         \PY{c+c1}{\PYZsh{} InceptionV3 utiliza un formato diferente, el rango de intensidades va de \PYZhy{}1 a 1. Vamos a reescalar este rango.}
         \PY{c+c1}{\PYZsh{} Pasamos de [0,256) \PYZhy{}\PYZgt{} [\PYZhy{}1,1]}
         
         \PY{n}{img\PYZus{}hacked} \PY{o}{/}\PY{o}{=} \PY{l+m+mi}{255} 
         \PY{n}{img\PYZus{}hacked} \PY{o}{\PYZhy{}}\PY{o}{=} \PY{l+m+mf}{0.5} 
         \PY{n}{img\PYZus{}hacked} \PY{o}{*}\PY{o}{=} \PY{l+m+mi}{2} 
         
         \PY{n}{img\PYZus{}hacked} \PY{o}{=} \PY{n}{img\PYZus{}hacked}\PY{o}{.}\PY{n}{reshape}\PY{p}{(}\PY{p}{[}\PY{l+m+mi}{1}\PY{p}{,} \PY{n}{img\PYZus{}hacked}\PY{o}{.}\PY{n}{shape}\PY{p}{[}\PY{l+m+mi}{0}\PY{p}{]}\PY{p}{,} \PY{n}{img\PYZus{}hacked}\PY{o}{.}\PY{n}{shape}\PY{p}{[}\PY{l+m+mi}{1}\PY{p}{]}\PY{p}{,} \PY{n}{img\PYZus{}hacked}\PY{o}{.}\PY{n}{shape}\PY{p}{[}\PY{l+m+mi}{2}\PY{p}{]}\PY{p}{]}\PY{p}{)}
         \PY{n}{y} \PY{o}{=} \PY{n}{iv3}\PY{o}{.}\PY{n}{predict}\PY{p}{(}\PY{n}{img\PYZus{}hacked}\PY{p}{)}
         
         \PY{n}{decode\PYZus{}predictions}\PY{p}{(}\PY{n}{y}\PY{p}{)}
\end{Verbatim}


\begin{Verbatim}[commandchars=\\\{\}]
{\color{outcolor}Out[{\color{outcolor}16}]:} [[('n07749582', 'lemon', 0.62572002),
           ('n04476259', 'tray', 0.026909433),
           ('n04522168', 'vase', 0.025146704),
           ('n07930864', 'cup', 0.024562402),
           ('n03950228', 'pitcher', 0.018331205)]]
\end{Verbatim}
            
    Efectivamente, el resultado dice que lo que ha recibido en la imagen es
un limón.

    \hypertarget{minimizar-la-diferencia-entre-imuxe1genes}{%
\subsubsection{Minimizar la diferencia entre
imágenes}\label{minimizar-la-diferencia-entre-imuxe1genes}}

Ahora se quiere decir a la red neuronal que no se extralimite haciendo
la manipulación, así la imagen pasará por una imagen normal:

    \begin{Verbatim}[commandchars=\\\{\}]
{\color{incolor}In [{\color{incolor} }]:} \PY{n}{adv} \PY{o}{=} \PY{n}{np}\PY{o}{.}\PY{n}{copy}\PY{p}{(}\PY{n}{X}\PY{p}{)}
        
        \PY{c+c1}{\PYZsh{} Límite de la perturbación}
        \PY{n}{pert} \PY{o}{=} \PY{l+m+mf}{0.01}
        
        \PY{n}{max\PYZus{}pert} \PY{o}{=} \PY{n}{X} \PY{o}{+} \PY{n}{pert}
        \PY{n}{min\PYZus{}pert} \PY{o}{=} \PY{n}{X} \PY{o}{\PYZhy{}} \PY{n}{pert}
        
        \PY{n}{cost} \PY{o}{=} \PY{l+m+mf}{0.0}
        
        \PY{k}{while} \PY{n}{cost} \PY{o}{\PYZlt{}} \PY{l+m+mf}{0.95}\PY{p}{:}
            \PY{n}{gr}\PY{p}{,} \PY{n}{cost} \PY{o}{=} \PY{n}{optimize\PYZus{}gradient}\PY{p}{(}\PY{p}{[}\PY{n}{adv}\PY{p}{,} \PY{l+m+mi}{0}\PY{p}{]}\PY{p}{)} \PY{c+c1}{\PYZsh{} Utilizamos la copia que realizamos anteriormente}
            \PY{n}{adv} \PY{o}{+}\PY{o}{=} \PY{n}{gr} \PY{c+c1}{\PYZsh{} Manipulamos los píxeles}
            \PY{n}{adv} \PY{o}{=} \PY{n}{np}\PY{o}{.}\PY{n}{clip}\PY{p}{(}\PY{n}{adv}\PY{p}{,} \PY{n}{min\PYZus{}pert}\PY{p}{,} \PY{n}{max\PYZus{}pert}\PY{p}{)} \PY{c+c1}{\PYZsh{} Ponemos el límite a la perturbación}
            \PY{n}{adv} \PY{o}{=} \PY{n}{np}\PY{o}{.}\PY{n}{clip}\PY{p}{(}\PY{n}{adv}\PY{p}{,} \PY{o}{\PYZhy{}}\PY{l+m+mi}{1}\PY{p}{,} \PY{l+m+mi}{1}\PY{p}{)} \PY{c+c1}{\PYZsh{} Ponemos el límite a la intensidad}
\end{Verbatim}


    Para mostrar el resultado, y ver que la imagen parece que no ha sido
alterada:

    \begin{Verbatim}[commandchars=\\\{\}]
{\color{incolor}In [{\color{incolor}18}]:} \PY{n}{adv} \PY{o}{/}\PY{o}{=} \PY{l+m+mi}{2}
         \PY{n}{adv} \PY{o}{+}\PY{o}{=} \PY{l+m+mf}{0.5}
         \PY{n}{adv} \PY{o}{*}\PY{o}{=} \PY{l+m+mi}{255}
\end{Verbatim}


    \begin{Verbatim}[commandchars=\\\{\}]
{\color{incolor}In [{\color{incolor}19}]:} \PY{n}{plt}\PY{o}{.}\PY{n}{imshow}\PY{p}{(}\PY{n}{adv}\PY{p}{[}\PY{l+m+mi}{0}\PY{p}{]}\PY{o}{.}\PY{n}{astype}\PY{p}{(}\PY{n}{np}\PY{o}{.}\PY{n}{uint8}\PY{p}{)}\PY{p}{)}
         \PY{n}{plt}\PY{o}{.}\PY{n}{show}\PY{p}{(}\PY{p}{)}
\end{Verbatim}


    \begin{center}
    \adjustimage{max size={0.9\linewidth}{0.9\paperheight}}{output_36_0.png}
    \end{center}
    { \hspace*{\fill} \\}
    
    Si realizamos la predicción de nuevo, veremos como nos dice que lo que
hay en la imagen es un limón:

    \begin{Verbatim}[commandchars=\\\{\}]
{\color{incolor}In [{\color{incolor}20}]:} \PY{n}{im} \PY{o}{=} \PY{n}{Image}\PY{o}{.}\PY{n}{fromarray}\PY{p}{(}\PY{n}{adv}\PY{p}{[}\PY{l+m+mi}{0}\PY{p}{]}\PY{o}{.}\PY{n}{astype}\PY{p}{(}\PY{n}{np}\PY{o}{.}\PY{n}{uint8}\PY{p}{)}\PY{p}{)}
         \PY{n}{im}\PY{o}{.}\PY{n}{save}\PY{p}{(}\PY{l+s+s2}{\PYZdq{}}\PY{l+s+s2}{./gato\PYZus{}hacked\PYZus{}good.png}\PY{l+s+s2}{\PYZdq{}}\PY{p}{)}
         
         \PY{c+c1}{\PYZsh{} Cargamos la imagen en una variable como array con un tamaño redimensionado a 299x299}
         \PY{n}{img\PYZus{}hacked} \PY{o}{=} \PY{n}{image}\PY{o}{.}\PY{n}{img\PYZus{}to\PYZus{}array}\PY{p}{(}\PY{n}{image}\PY{o}{.}\PY{n}{load\PYZus{}img}\PY{p}{(}\PY{l+s+s2}{\PYZdq{}}\PY{l+s+s2}{./gato\PYZus{}hacked\PYZus{}good.png}\PY{l+s+s2}{\PYZdq{}}\PY{p}{)}\PY{p}{)} 
         
         \PY{c+c1}{\PYZsh{} InceptionV3 utiliza un formato diferente, el rango de intensidades va de \PYZhy{}1 a 1. Vamos a reescalar este rango.}
         \PY{c+c1}{\PYZsh{} Pasamos de [0,256) \PYZhy{}\PYZgt{} [\PYZhy{}1,1]}
         
         \PY{n}{img\PYZus{}hacked} \PY{o}{/}\PY{o}{=} \PY{l+m+mi}{255} 
         \PY{n}{img\PYZus{}hacked} \PY{o}{\PYZhy{}}\PY{o}{=} \PY{l+m+mf}{0.5} 
         \PY{n}{img\PYZus{}hacked} \PY{o}{*}\PY{o}{=} \PY{l+m+mi}{2} 
         
         \PY{n}{img\PYZus{}hacked} \PY{o}{=} \PY{n}{img\PYZus{}hacked}\PY{o}{.}\PY{n}{reshape}\PY{p}{(}\PY{p}{[}\PY{l+m+mi}{1}\PY{p}{,} \PY{n}{img\PYZus{}hacked}\PY{o}{.}\PY{n}{shape}\PY{p}{[}\PY{l+m+mi}{0}\PY{p}{]}\PY{p}{,} \PY{n}{img\PYZus{}hacked}\PY{o}{.}\PY{n}{shape}\PY{p}{[}\PY{l+m+mi}{1}\PY{p}{]}\PY{p}{,} \PY{n}{img\PYZus{}hacked}\PY{o}{.}\PY{n}{shape}\PY{p}{[}\PY{l+m+mi}{2}\PY{p}{]}\PY{p}{]}\PY{p}{)}
         \PY{n}{y} \PY{o}{=} \PY{n}{iv3}\PY{o}{.}\PY{n}{predict}\PY{p}{(}\PY{n}{img\PYZus{}hacked}\PY{p}{)}
         
         \PY{n}{decode\PYZus{}predictions}\PY{p}{(}\PY{n}{y}\PY{p}{)}
\end{Verbatim}


\begin{Verbatim}[commandchars=\\\{\}]
{\color{outcolor}Out[{\color{outcolor}20}]:} [[('n07749582', 'lemon', 0.99910444),
           ('n07747607', 'orange', 0.00026996125),
           ('n09229709', 'bubble', 5.943119e-05),
           ('n07753275', 'pineapple', 4.1333893e-05),
           ('n07718472', 'cucumber', 3.5098896e-05)]]
\end{Verbatim}
            
    Finalmente, se obtiene como resultado que lo que tiene la imagen es un
limón (a pesar de ser un gato), y la imagen ya no parece modificada.


    % Add a bibliography block to the postdoc
    
    
    
    \end{document}
